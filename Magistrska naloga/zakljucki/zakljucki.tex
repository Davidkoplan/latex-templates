% !TeX spellcheck = si_SI
\chapter{Zaključki}\label{cha:zakljucki}

V zaključku opišite glavne rezultate in ugotovitve, ki jih povzamete v nekaj (oštevilčenih) točkah. Pazite, da zaključek ne bo ponovitev uvoda. Tukaj opišite oz. povzemite izključno tisto, kar je bilo narejeno in ugotovljeno:
\begin{enumerate}
\item Izmerili smo / Zasnovali smo \ldots
\item Pokazali smo \ldots
\item Dobljeni rezultati pomenijo \ldots
\item Ugotovili smo \ldots
\item \ldots
\item \ldots
\end{enumerate}

Na koncu na kratko (v največ 5 vrsticah) zapišite celovit doprinos dela na osnovi opisanih zaključkov.

\textbf{Predlogi za nadaljnje delo}

V posebnem odstavku napišite predloge za nadaljnje delo na obravnavanem področju.
