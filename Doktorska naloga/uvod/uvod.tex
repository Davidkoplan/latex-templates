% !TeX spellcheck = si_SI
\chapter{Uvod}\label{cha:uvod}

\section{Obravnavano znanstveno področje in opredelitev problema}\label{sec:predstavitev_problema}
Na tem mestu na podlagi predstavljenega \emph{Stanja razvoja} povzemite dosedanje ključne izsledke in ugotovitve, predvsem pa odprta vprašanja in problematiko, ki do tega trenutka še niso bili razrešeni.

\section{Pregled stanja}\label{sec:pregled_stanja}
V tem poglavju predstavite celovit pregled dela raziskovalcev na obravnavanem področju, ki ste ga predhodno definirali v poglavju \ref{sec:predstavitev_problema}. Ta pregled vam bo v nadaljevanju služil kot opora za argumentacijo znanstvenega doprinosa vašega dela, obenem pa bo v zgoščeni obliki seznanil bralca z dosedanjim raziskovalnim delom na obravnavanem področji, tako kronološko kot tudi vsebinsko.

\section{Teza in cilji}\label{sec:teza}
V tem poglavju predstavite raziskovalno hipotezo, ki ste jo postavili ob pregledu \emph{Stanja razvoja} in vam je služila kot izhodišče za zasnovo doktorskega dela. Predstavite konkretne cilje, ki jih boste v povezavi z raziskovalno hipotezo v okviru doktorskega dela skušali doseči in ki bodo torej neposredno predstavljali prispevek znanosti ali tudi širše. Pišite, kaj boste delali, kaj pričakujete od teoretičnih in kaj od praktičnih raziskav ter kakšna so tveganja in nevarnosti, da teh ciljev ne boste dosegli. Pišite o hipotezah in ne o dobljenih rezultatih.

\section{Potek dela}\label{sec:potek}
V tem delu se osredotočite na to, kako je delo strukturirano (po poglavjih), torej npr. kje je predstavljen nek vsebinski sklop in kako so posamezni vsebinski sklopi med seboj povezani.

\section{Publikacije}\label{sec:publikacije}
V primeru, da so bili rezultati doktorskega dela objavljeni ali pripravljeni kot celovite znanstvene publikacije, lahko v tem podpoglavju navedete seznam objavljenih publikacij. Predstavitev publikacij (seznam) je prostovoljna rubrika.





